% Options for packages loaded elsewhere
\PassOptionsToPackage{unicode}{hyperref}
\PassOptionsToPackage{hyphens}{url}
%
\documentclass[
]{article}
\usepackage{amsmath,amssymb}
\usepackage{lmodern}
\usepackage{iftex}
\ifPDFTeX
  \usepackage[T1]{fontenc}
  \usepackage[utf8]{inputenc}
  \usepackage{textcomp} % provide euro and other symbols
\else % if luatex or xetex
  \usepackage{unicode-math}
  \defaultfontfeatures{Scale=MatchLowercase}
  \defaultfontfeatures[\rmfamily]{Ligatures=TeX,Scale=1}
\fi
% Use upquote if available, for straight quotes in verbatim environments
\IfFileExists{upquote.sty}{\usepackage{upquote}}{}
\IfFileExists{microtype.sty}{% use microtype if available
  \usepackage[]{microtype}
  \UseMicrotypeSet[protrusion]{basicmath} % disable protrusion for tt fonts
}{}
\makeatletter
\@ifundefined{KOMAClassName}{% if non-KOMA class
  \IfFileExists{parskip.sty}{%
    \usepackage{parskip}
  }{% else
    \setlength{\parindent}{0pt}
    \setlength{\parskip}{6pt plus 2pt minus 1pt}}
}{% if KOMA class
  \KOMAoptions{parskip=half}}
\makeatother
\usepackage{xcolor}
\usepackage[left=2cm,right=2cm,top=2cm,bottom=2cm]{geometry}
\usepackage{listings}
\newcommand{\passthrough}[1]{#1}
\lstset{defaultdialect=[5.3]Lua}
\lstset{defaultdialect=[x86masm]Assembler}
\usepackage{longtable,booktabs,array}
\usepackage{calc} % for calculating minipage widths
% Correct order of tables after \paragraph or \subparagraph
\usepackage{etoolbox}
\makeatletter
\patchcmd\longtable{\par}{\if@noskipsec\mbox{}\fi\par}{}{}
\makeatother
% Allow footnotes in longtable head/foot
\IfFileExists{footnotehyper.sty}{\usepackage{footnotehyper}}{\usepackage{footnote}}
\makesavenoteenv{longtable}
\usepackage{graphicx}
\makeatletter
\def\maxwidth{\ifdim\Gin@nat@width>\linewidth\linewidth\else\Gin@nat@width\fi}
\def\maxheight{\ifdim\Gin@nat@height>\textheight\textheight\else\Gin@nat@height\fi}
\makeatother
% Scale images if necessary, so that they will not overflow the page
% margins by default, and it is still possible to overwrite the defaults
% using explicit options in \includegraphics[width, height, ...]{}
\setkeys{Gin}{width=\maxwidth,height=\maxheight,keepaspectratio}
% Set default figure placement to htbp
\makeatletter
\def\fps@figure{htbp}
\makeatother
\usepackage[normalem]{ulem}
\setlength{\emergencystretch}{3em} % prevent overfull lines
\providecommand{\tightlist}{%
  \setlength{\itemsep}{0pt}\setlength{\parskip}{0pt}}
\setcounter{secnumdepth}{-\maxdimen} % remove section numbering
\usepackage{graphicx}
\usepackage{float}
\usepackage{amsmath}
\nocite{*}
\usepackage[numbers]{natbib}
\usepackage{titlesec}
\titleformat{\section}[block]{\color{black}\Large\bfseries\filcenter}{}{1em}{}
\usepackage{mdframed}
\usepackage{booktabs}
\usepackage{longtable}
\usepackage{array}
\usepackage{multirow}
\usepackage{wrapfig}
\usepackage{float}
\usepackage{colortbl}
\usepackage{pdflscape}
\usepackage{tabu}
\usepackage{threeparttable}
\usepackage{threeparttablex}
\usepackage[normalem]{ulem}
\usepackage{makecell}
\usepackage{xcolor}
\ifLuaTeX
  \usepackage{selnolig}  % disable illegal ligatures
\fi
\IfFileExists{bookmark.sty}{\usepackage{bookmark}}{\usepackage{hyperref}}
\IfFileExists{xurl.sty}{\usepackage{xurl}}{} % add URL line breaks if available
\urlstyle{same} % disable monospaced font for URLs
\hypersetup{
  pdftitle={DESIGN AND ANALYSIS OF EXPERIMENTS},
  pdfauthor={B.M Njuguna},
  hidelinks,
  pdfcreator={LaTeX via pandoc}}

\title{\textbf{DESIGN AND ANALYSIS OF EXPERIMENTS}}
\author{\textbf{B.M Njuguna}}
\date{\textbf{2022-10-29}}

\begin{document}
\maketitle

\newpage
\tableofcontents
\newpage

\hypertarget{introduction}{%
\section{Introduction}\label{introduction}}

This paper is based on advanced experimental design for scientific
Studies. The various formulas used in different designs and experiments
are as outlined below.

\hypertarget{one-way-classification-model}{%
\section{1. One Way Classification
Model}\label{one-way-classification-model}}

This is basically the Analysis of Covariance(ANCOVA). Here single factor
experiment with single covariate is considered. The model is as follows;

\[Y_{ij} = \mu+\tau_i+\beta(x_{ij}-\bar{x..} )+ \epsilon_{ij}\]

where; i = 1, 2, \ldots, t and j = 1,2, \ldots{} r

\(Y_{ij}\)- is the jth response variable taken under the ith treatment

\(\mu\) - overall mean

\(x_{ij}\) - The measure of covariate corresponding to \(Y_{ij}\)

\(\bar{x..}\) - The mean of \(x_{ij}\) value

\(\beta\) - The linear regression coefficient of \(Y_{ij}\) and
\(x{ij}\)

The notations used are;

\[S_{yy}=\sum_{i=1}^t\sum_{j=1}^r(Y_{ij}-\bar{Y..})^2=\sum_{i=1}^t\sum_{j=1}^rY_{ij}^2-\frac{Y..^2}{tr}\]

\[S_{xx}=\sum_{i=1}^t\sum_{j=1}^r(X_{ij}-\bar{x..})^2=\sum_{i=1}^t\sum_{j=1}^rX_{ij}^2-\frac{X..^2}{tr}\]

\[S_{xy}=\sum_{i=1}^t\sum_{j=1}^r(Y_{ij}-\bar{Y..})(X_{ij}-\bar{x..})=\sum_{i=1}^t\sum_{j=1}^rY_{ij}X_{ij}-\frac{Y..X..}{tr}\]

\[T_{yy}=\sum_{i=1}^t\sum_{j=1}^r(Y_{i.}-\bar{Y..})^2=\frac{\sum_{i=1}^tY_{i.}^2}{r}-\frac{Y_{..}^2}{tr}\]

\[T_{xx}=\sum_{i=1}^t\sum_{j=1}^r(X_{i.}-\bar{X..})^2=\frac{\sum_{i=1}^tX_{i.}^2}{r}-\frac{X_{..}^2}{tr}\]

\[\frac{\sum_{i=1}^tY_{i}Xi}{r}-\frac{Y_{..}X..}{tr}\]

\[E_{yy}=S_{yy}-T_{yy}\]

\[E_{xx}=S_{xx}-T_{xx}\]

\[E_{xy}=S_{xy}-T_{xy}\]

The statistical analysis is;

LSE of \(\mu\) is; \(\hat{\mu}=\bar{Y..}\)

Then;

\[\hat{\beta}=\frac{E_{xy}}{E_{xx}}\]

\[SSE = E_{yy} - \frac{E_{xy}^2}{Exx}\]

SSE usually have t(r-1)-1 degrees of freedom

Now suppose we use to test \(\tau_i=0\). Then under Null hypothesis, the
reduced model will be;

\[Y_{ij} = \mu+\beta(x_{ij}-\bar{x..} + \epsilon_{y}\]

Then;

\[\hat{\beta}=\frac{S_{xy}}{S_{xx}}\]

And;

\[SSE' = S_{yy}-\frac{S_{xy}^2}{S_{xx}}\]

note \(SSE\) is smaller than \(SSE'\), where \(SSE'-SSE\) is a reduction
in sums of squares due to \(\tau_i\). Therefore for testing
\(\tau_i=0\), the test statistic is;

\[F_{calc}=\frac{(SSE'-SSE)/(t-1)}{SSE/(t(r-1)-1)}\]

We test it against \(F_{(t-1),t(r-1)-1,\alpha}\). The ANOVA table is as
follows;

\begin{tabular}{l|l|l|l|l|l|l}
\hline
...1 & ...2 & SS and Product & ...4 & ...5 & Adjustment for Regression & ...7\\
\hline
Source of Variation & df & X & XY & Y & Y & df\\
\hline
Treatment & t-1 & Txx & Txy & Tyy & NA & NA\\
\hline
Error & t(r-1) & Exx & Exy & Eyy & SSE & t(r-1)-1\\
\hline
Total & tr-1 & Sxx & Sxy & Syy & SSE' & tr-2\\
\hline
Adjusted Treatment & NA & NA & NA & NA & SSE'-SSE & t-1\\
\hline
\end{tabular}

\newpage

\hypertarget{nested-designs}{%
\section{2. Nested Designs}\label{nested-designs}}

\hypertarget{i.-two---stage-nested-designs}{%
\subsection{i. Two - Stage Nested
Designs}\label{i.-two---stage-nested-designs}}

The statistical model is;

\[Y_{ijk}=\mu+\tau_i+\beta_{j(i)}+\epsilon_{(ij)k}\]

where, i = 1,2,\ldots,a, j = 1,2,\ldots,b, and k = i,2,\ldots r

\(\mu\) - overall mean

\(\tau_i\) - Effect of the ith factor A

\(\beta_{j(i)}\) - Effect of the jth factor B nested under factor A

\(\epsilon_{(ij)k}\) - Random error term.

The sums of squares are partitioned as follows;

\[SS_{total} = SS_{A}+ss_{B(A)}+SS_{Error}\]

They are calculated as follows;

\[SS_{Total} = \sum_{i=1}^a\sum_{j=1}^b\sum_{k=1}^rY_{ijk}^2-\frac{(Y...)^2}{abr}\]

\[SS_A=\frac{\sum_{i=1}^aY_{i..}^2}{br}-\frac{(Y...)^2}{abr}\]

\[SS_{B(A)} = \frac{\sum_{i=1}^a\sum_{j=1}^bY_{ij.}^2}{r}-\frac{\sum_{i=1}^aYi..^2}{br}\]

\[SS_{Error}=\sum_{i=1}^a\sum_{j=1}^b\sum_{k=1}^rY_{ijk}^2-\frac{\sum_{i=1}^a\sum_{j=1}^bY_{ij.}^2}{r}\]

The ANOVA table is as follows;

\begin{longtable}[]{@{}
  >{\centering\arraybackslash}p{(\columnwidth - 6\tabcolsep) * \real{0.1264}}
  >{\centering\arraybackslash}p{(\columnwidth - 6\tabcolsep) * \real{0.2011}}
  >{\centering\arraybackslash}p{(\columnwidth - 6\tabcolsep) * \real{0.2931}}
  >{\centering\arraybackslash}p{(\columnwidth - 6\tabcolsep) * \real{0.3678}}@{}}
\toprule()
\begin{minipage}[b]{\linewidth}\centering
Source of Variation
\end{minipage} & \begin{minipage}[b]{\linewidth}\centering
df
\end{minipage} & \begin{minipage}[b]{\linewidth}\centering
SS
\end{minipage} & \begin{minipage}[b]{\linewidth}\centering
MS
\end{minipage} \\
\midrule()
\endhead
\(
           A
          \) & \(
                            a-1
                            \) & \(
                                        SS_{A}
                                         \) & \(
                                                      MS_A
                                                      \) \\
\(
         B(A)
          \) & \(a(b-1)\) & \(
                                      SS_{B(A)}
                                         \) & \(
                                                   MS_{B(A)}
                                                      \) \\
\(
         Error
          \) & \(ab(r-1)\) & \(
                                      SS_{Error}
                                         \) & \(
                                                   MS_{Error}
                                                      \) \\
\(
         Total
          \) & \(
                           abr-1
                            \) & \(
                                      SS_{Total}
                                         \) & \\
\bottomrule()
\end{longtable}

The appropriate statistic for testing the effect of factor A and B
depends on whether the levels of A and B are fixed or random. The
Expected Mean Squares in the two stage nested design are as follows;

\begin{longtable}[]{@{}
  >{\raggedright\arraybackslash}p{(\columnwidth - 6\tabcolsep) * \real{0.0838}}
  >{\centering\arraybackslash}p{(\columnwidth - 6\tabcolsep) * \real{0.3353}}
  >{\centering\arraybackslash}p{(\columnwidth - 6\tabcolsep) * \real{0.3114}}
  >{\centering\arraybackslash}p{(\columnwidth - 6\tabcolsep) * \real{0.2575}}@{}}
\toprule()
\begin{minipage}[b]{\linewidth}\raggedright
\[
E(MS)
\]
\end{minipage} & \begin{minipage}[b]{\linewidth}\centering
A - fixed

B- fixed
\end{minipage} & \begin{minipage}[b]{\linewidth}\centering
A - fixed

B - Random
\end{minipage} & \begin{minipage}[b]{\linewidth}\centering
A - Random

B - Random
\end{minipage} \\
\midrule()
\endhead
\(
E(MS_A)
\) & \(
\frac{\sigma^2+br\sum\tau_i^2}{a-1}
\) & \(
\sigma^2+r\sigma^2_\beta+\frac{br\sum\tau_i}{a-1}
\) & \(
\sigma^2+r\sigma^2_\beta+br\sigma^2_\tau
\) \\
\(
E(MS_{AB})
\) & \(
\sigma^2+\frac{r\sum_{i}\sum_j\beta_{j(i)}^2}{a(b-1)}
\) & \(
\sigma^2+r\sigma^2_\beta
\) & \(
\sigma^2+r\sigma^2_\beta
\) \\
\(
E(MS_E)
\) & \(
\sigma^2
\) & \(
\sigma^2
\) & \(
\sigma^2
\) \\
\bottomrule()
\end{longtable}

The testing of hypothesis is as follows;

The testing of hypothesis is as follows;

\begin{enumerate}
\def\labelenumi{\arabic{enumi}.}
\tightlist
\item
  \uline{\textbf{when A is fixed and B is random}}
\end{enumerate}

To test \(H_0:\tau_i=0\) vs \(H_1:\tau_i\ne0\) the test statistic is;

\[F_{calc}=\frac{MS_A}{MS_{B(A)}}\]

We reject \(H_0\) is \(F_{calc}>F_{a-1,a(b-1),\alpha}\)

To test \(H_0:\sigma^2_\beta=0\) vs \(H_1:\sigma^2_\beta\neq0\), the
test statistic is;

\[F{(calc)=\frac{MS_{B(A)}}{MS_{Error}}}\]

We reject \(H_0\) is \(F_{calc}>F_{a(b-1),ab(r-1),\alpha}\)

\begin{enumerate}
\def\labelenumi{\arabic{enumi}.}
\setcounter{enumi}{1}
\tightlist
\item
  \uline{\textbf{A and B fixed}}
\end{enumerate}

To test \(H_0:\tau_i=0\) vs \(H_1:\tau_i\ne0\) the test statistic is;

\[F_{calc}=\frac{MS_A}{MS_{Error}}\]

To test \(H_0:\beta_{j(i)}=0\) vs \(H_1:\beta_{j(i)}\neq0\), the test
statistic is;

\[F{(calc)=\frac{MS_{B(A)}}{MS_{Error}}}\]

\begin{enumerate}
\def\labelenumi{\arabic{enumi}.}
\setcounter{enumi}{2}
\tightlist
\item
  \uline{\textbf{A and B random}}
\end{enumerate}

To test \(H_0:\sigma^2_{\tau}=0\) vs \(H_1:\sigma^2{_\tau}\ne0\) the
test statistic is;

\[F_{calc}=\frac{MS_A}{MS_{B(A)}}\]

We reject \(H_0\) is \(F_{calc}>F_{a-1,a(b-1),\alpha}\)

To test \(H_0:\sigma^2_\beta=0\) vs \(H_1:\sigma^2_\beta\neq0\), the
test statistic is;

\[F{(calc)=\frac{MS_{B(A)}}{MS_{Error}}}\]

We reject \(H_0\) is \(F_{calc}>F_{a(b-1),ab(r-1),\alpha}\)

\hypertarget{ii.-three-stage-nested-design}{%
\subsection{ii. Three Stage Nested
Design}\label{ii.-three-stage-nested-design}}

The model is written as;

\[Y_{ijkl}=\mu+\tau_i+\beta_{j(i)}+\gamma_{k(ij)}+\epsilon_{(ijk)l}\]

where; i=1,2,\ldots a, j = 1,2,\ldots,b, k = 1,2,\ldots c and l =
1,2,\ldots r

\(\mu\) - overall mean

\(\tau_i\) - Effect of the ith factor A

\(\beta_{j(i)}\) - Effect of the jth factor B nested under factor A

\(\gamma_{k(ij)}\) - Effect of the kth factor B nested under factor A
and B

\(\epsilon_{(ij)k}\) - Random error term.

The sums of squares are partitioned as follows;

\[SS_{Total}=SS_A+SS_{B(A)}+SS_{C(B)}+SS_{Error}\]

\[SS_{Total = \sum_i\sum_j\sum_k\sum_lY_{ijkl}^2}-\frac{Y_{...}^2}{abcr}\]

\[SS_A = \frac{\sum_iY_{i...}^2}{bcr}-CT\]

\[SS_{B(A)}=\frac{\sum_i\sum_jY_{ij..}^2}{cr}-\frac{\sum_iY_{i...}^2}{bcr}\]

\[SS_{C(B)}=\frac{\sum_i\sum_j\sum_kY_{ijk.}^2}{r}-\frac{\sum_i\sum_jY_{ij..}^2}{cr}\]

\[SS_{Error}=\sum_i\sum_j\sum_k\sum_lY_{ijkl}^2-\frac{\sum_i\sum_j\sum_kY_{ijk.}^2}{r}\]

The ANOVA Table is as follows;

\begin{longtable}[]{@{}
  >{\centering\arraybackslash}p{(\columnwidth - 6\tabcolsep) * \real{0.1310}}
  >{\centering\arraybackslash}p{(\columnwidth - 6\tabcolsep) * \real{0.2083}}
  >{\centering\arraybackslash}p{(\columnwidth - 6\tabcolsep) * \real{0.2857}}
  >{\centering\arraybackslash}p{(\columnwidth - 6\tabcolsep) * \real{0.3631}}@{}}
\toprule()
\begin{minipage}[b]{\linewidth}\centering
Source of Variation
\end{minipage} & \begin{minipage}[b]{\linewidth}\centering
df
\end{minipage} & \begin{minipage}[b]{\linewidth}\centering
SS
\end{minipage} & \begin{minipage}[b]{\linewidth}\centering
MS
\end{minipage} \\
\midrule()
\endhead
\(
           A
          \) & \(
                          a-1
                          \) & \(
                                      SS_A
                                      \) & \(
                                                   MS_A
                                                   \) \\
\(
         B(A)
          \) & \(
                          b-1
                          \) & \(
                                   SS_{B(A)}
                                      \) & \(
                                                MS_{B(A)}
                                                   \) \\
\(
         C(A)
          \) & \(
                        ab(c-1)
                          \) & \(
                                   SS_{C(B)}
                                      \) & \(
                                                MS_{C(B)}
                                                   \) \\
\(
         Error
          \) & \(
                        abc(r-1)
                          \) & \(
                                   SS_{Error}
                                      \) & \(
                                                MS_{Error}
                                                   \) \\
\(
         Total
          \) & \(
                         abcr-1
                          \) & \(
                                   SS_{Total}
                                      \) & \\
\bottomrule()
\end{longtable}

\newpage

\hypertarget{balanced-incomplete-block-design}{%
\section{3. Balanced Incomplete Block
Design}\label{balanced-incomplete-block-design}}

In some experiments, the number of treatments is large, the adoption of
RCBD therefore may result in an increase of error variance due to large
block size. Here we use the BIBD. The parameters of BIBD are
\(t, b, r, k, \lambda\). For an experiment to be a BIBD, it must satisfy
the following two conditions;

\begin{enumerate}
\def\labelenumi{\arabic{enumi}.}
\item
  \(bk=rt\)
\item
  \(\lambda(t-1)=r(k-1)\), this implies that;
  \(\lambda=\frac{r(k-1)}{t-1}\). Note that \(\lambda\) is the number of
  times each pair of treatment appear or occur together in the same
  block.
\end{enumerate}

The model can be written as;

\[Y_{ij}=\mu+\tau_i+\beta_j+\epsilon_{ij}\]

where; i = 1,2,\ldots,t, j = 1,2,\ldots,r

\(\mu\) - overall mean

\(\tau_i\) - Effect of the ith treatment

\(\beta_j\) - Effect of the jth block

\(\epsilon_{ij}\) - Random error term

The sums of squares are partitioned as follows;

\[SS_{Total}=SS_{Block}+SS_{treat(adj)}+SS_{Error}\]

\[SS_{Total = \sum_i\sum_jY_{ij}^2}- \frac{Y_{..}^2}{N=(r\times t\space or\space b\times k)}\]

\[SS_{Block}=\frac{\sum_{j=1}^kY_{ij}^2}{k}-\frac{Y_{..}^2}{N}\]

\[SS_{treat(adj)}=k\frac{\sum_{i=1}^rQ_I}{\lambda t}\]

where \(Q_i\) is the adjusted total for the ith treatment computed as;

\[Q_i=Y_{i.}-\frac{1}{k}\sum_jn_{ijY_{.j}}\]

i = 1,2,\ldots,t

Then; \(n_{ij}=1\) if treatment \(i\) appears in block \(j\) and
\(n_{ij}=0\) otherwise.

The ANOVA table is as follows;

\begin{longtable}[]{@{}
  >{\centering\arraybackslash}p{(\columnwidth - 6\tabcolsep) * \real{0.1344}}
  >{\centering\arraybackslash}p{(\columnwidth - 6\tabcolsep) * \real{0.1882}}
  >{\centering\arraybackslash}p{(\columnwidth - 6\tabcolsep) * \real{0.2849}}
  >{\centering\arraybackslash}p{(\columnwidth - 6\tabcolsep) * \real{0.3817}}@{}}
\toprule()
\begin{minipage}[b]{\linewidth}\centering
Source of Variation
\end{minipage} & \begin{minipage}[b]{\linewidth}\centering
df
\end{minipage} & \begin{minipage}[b]{\linewidth}\centering
SS
\end{minipage} & \begin{minipage}[b]{\linewidth}\centering
MS
\end{minipage} \\
\midrule()
\endhead
\(
         Block
          \) & \(
                          b-1
                          \) & \(
                                    SS_{Block}
                                        \) & \(
                                                      MS_{Block}
                                                          \) \\
\(
      Treat(adj)
          \) & \(
                          t-1
                          \) & \(
                                  SS_{treat(adj)}
                                        \) & \(
                                                    MS_{treat(adj)}
                                                          \) \\
\(
         Error
          \) & \(
                        N-t-b+1
                          \) & \(
                                    SS_{Error}
                                        \) & \(
                                                      MS_{Error}
                                                          \) \\
\(
         Total
          \) & \(
                          N-1
                          \) & \(
                                    SS_{Total}
                                        \) & \\
\bottomrule()
\end{longtable}

To test \(H_o:\tau_i=0\) vs \(H_1:\tau_i\ne0\), the test statistic is;

\[F_{calc}=\frac{MS_{treat(adj)}}{MS_{Error}}\]

We reject \(H_0\) if \(F_{calc}>F_{t-1,N-t-b+1}\)

\newpage

\hypertarget{partially-balanced-incomplete-design.}{%
\section{4. Partially Balanced Incomplete
Design.}\label{partially-balanced-incomplete-design.}}

BIBD do not exist for all combination of parameters you might with to
employ. PBIBD are designs whereby the number of times a given treatment
occur will vary from treatment to treatment where some pairs of
treatment appear together \(\lambda_1\) times and others appear
\(\lambda_2\) times. Note \(\lambda_1>\lambda_2\). This is the simplest
of the PBIBD which had two associate classes; \(\lambda_1\) and
\(\lambda_2\). For a design to be PBIBD, it must satisfy the following
two conditions;

\[\sum_{i=1}^2n_i=t-1,\space n_1+n_2=t-1\]

\[\sum_{i=1}^2\lambda_in_i=r(k-1), \space \lambda_1n_1+\lambda_2n_2=r(k-1)\]

How to determine \(P_{jk}^1\), you pick any two pair of treatments that
are of first associate. Then to determine \(P_{jk}^2\), you pick any two
pairs of treatments which are second associate.

\hypertarget{lattice-designs}{%
\subsection{Lattice Designs}\label{lattice-designs}}

Consider a balanced incomplete design with \(t=k^2\) treatments arranged
in \(b=k(k+1)\) blocks with \(k\) runs per blocks and \(r=k+1\)
replicates. Such a design is called a \textbf{balanced lattice}.

\begin{enumerate}
\def\labelenumi{\arabic{enumi}.}
\item
  A design for \(k^2\) treatments in \(2k\) blocks of \(k\) runs with 2
  replicates is called a \textbf{simple lattice}.
\item
  A lattice design with \(t=k^2\) treatments in \(3k\) blocks grouped
  into 3 replicates, \(r=3\) is called a \textbf{triple lattice design}.
\item
  A lattice design for \(t = k^2\) treatments in \(4k\) blocks arranged
  in 4 replicates, \(r=4\) is called a \textbf{quadruple lattice}.
\item
  A lattice design for \(t=k^3\) treatments in \(k^2\) blocks of k runs
  is called a \textbf{cubic lattice design}.
\item
  A lattice design for \(t=k(k+1)\) treatments in \(k+1\) blocks of size
  k is called a \textbf{rectangular lattice design}.
\end{enumerate}

\end{document}
